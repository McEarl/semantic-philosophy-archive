\documentclass{stex}
\libinput{preamble}
\begin{document}
\usemodule{glossary/persons?epicurus}
\usemodule{glossary/persons?zeus}

\begin{sparagraph}[title=4]
  All bodily suffering is negligible: for that which causes acute pain has short duration, and that which endures long in the flesh causes but mild pain.
\end{sparagraph}

\vspace{0.5em}
\begin{sparagraph}[title=7]
  It is hard for an evil-doer to escape detection, but to obtain security for escaping is impossible.
\end{sparagraph}

\vspace{0.5em}
\begin{sparagraph}[title=9]
  Necessity is an evil, but there is no necessity to live under the control of necessity.
\end{sparagraph}

\vspace{0.5em}
\begin{sparagraph}[title=10]
  Remember that you are of mortal nature and have a limited time to live and have devoted yourself to discussions on nature for all time and eternity and have seen things that are now and are to come and have been.
\end{sparagraph}

\vspace{0.5em}
\begin{sparagraph}[title=11]
  For most men rest is stagnation and activity madness.
\end{sparagraph}

\vspace{0.5em}
\begin{sparagraph}[title=14]
  We are born once and cannot be born twice, but for all time must be no more.
  But you, who are not \textins{master} of tomorrow, postpone your happiness: life is wasted in procrastination and each one of us dies without allowing himself leisure.
\end{sparagraph}

\vspace{0.5em}
\begin{sparagraph}[title=15]
  We value our characters as something peculiar to ourselves, whether they are good and we are esteemed by men, or not; so ought we to value the characters of others, if they are well-disposed to us.
\end{sparagraph}

\vspace{0.5em}
\begin{sparagraph}[title=16]
  No one when he sees evil deliberately chooses it, but is enticed by it as being good in comparison with a greater evil and so pursues it.
\end{sparagraph}

\vspace{0.5em}
\begin{sparagraph}[title=17]
  It is not the young man who should be thought happy, but an old man who has lived a good life.
  For the young man at the height of his powers is unstable and is carried this way and that by fortune, like a headlong stream.
  But the old man has come to anchor in old age as though in port, and the good things for which before he hardly hoped he has brought into safe harbourage in his grateful recollections.
\end{sparagraph}

\vspace{0.5em}
\begin{sparagraph}[title=18]
  Remove sight, association and contact, and the passion of love is at an end.
\end{sparagraph}

\vspace{0.5em}
\begin{sparagraph}[title=19]
  Forgetting the good that has been he has become old this very day.
\end{sparagraph}

\vspace{0.5em}
\begin{sparagraph}[title=21]
  We must not violate nature, but obey her; and we shall obey her if we fulfil the necessary desires and also the physical, if they bring no harm to us, but sternly reject the harmful.
\end{sparagraph}

\vspace{0.5em}
\begin{sparagraph}[title=23]
  All friendship is desirable in itself, though it starts from the need of help.
\end{sparagraph}

\vspace{0.5em}
\begin{sparagraph}[title=24]
  Dreams have no divine character nor any prophetic force, but they originate from the influx of images.
\end{sparagraph}

\vspace{0.5em}
\begin{sparagraph}[title=25]
  Poverty, when measured by the natural purpose of life, is great wealth, but unlimited wealth is great poverty.
\end{sparagraph}

\vspace{0.5em}
\begin{sparagraph}[title=26]
  You must understand that whether the discourse be long or short it tends to the same end.
\end{sparagraph}

\vspace{0.5em}
\begin{sparagraph}[title=27]
  In all other occupations the fruit comes painfully after completion, but in philosophy pleasure goes hand in hand with knowledge; for enjoyment does not follow comprehension, but comprehension and enjoyment are simultaneous.
\end{sparagraph}

\vspace{0.5em}
\begin{sparagraph}[title=28]
  We must not approve either those who are always ready for friendship, or those who hang back, but for friendship’s sake we must even run risks.
\end{sparagraph}

\vspace{0.5em}
\begin{sparagraph}[title=29]
  In investigating nature I would prefer to speak openly and like an oracle to give answers serviceable to all mankind, even though no one should understand me, rather than to conform to popular opinions and so win the praise freely scattered by the mob.
\end{sparagraph}

\vspace{0.5em}
\begin{sparagraph}[title=30]
  Some men throughout their lives gather together the means of life, for they do not see that the draught swallowed by all of us at birth is a draught of death.
\end{sparagraph}

\vspace{0.5em}
\begin{sparagraph}[title=31]
  Against all else it is possible to provide security, but as against death all of us mortals alike dwell in an unfortified city.
\end{sparagraph}

\vspace{0.5em}
\begin{sparagraph}[title=32]
  The veneration of the wise man is a great blessing to those who venerate him.
\end{sparagraph}

\vspace{0.5em}
\begin{sparagraph}[title=33]
  The flesh cries out to be saved from hunger, thirst and cold.
  For if a man possess this safety and hope to possess it, he might rival even \sn{Zeus} in happiness.
\end{sparagraph}

\vspace{0.5em}
\begin{sparagraph}[title=34]
  It is not so much our friends’ help that helps us as the confidence of their help.
\end{sparagraph}

\vspace{0.5em}
\begin{sparagraph}[title=35]
  We should not spoil what we have by desiring what we have not, but remember that what we have too was the gift of fortune.
\end{sparagraph}

\vspace{0.5em}
\begin{sparagraph}[title=36]
  \sn{Epicurus}’ life when compared to other men’s in respect of gentleness and self-sufficiency might be thought a mere legend.
\end{sparagraph}

\vspace{0.5em}
\begin{sparagraph}[title=37]
  Nature is weak towards evil, not towards good: because it is saved by pleasures, but destroyed by pains.
\end{sparagraph}

\vspace{0.5em}
\begin{sparagraph}[title=38]
  He is a little man in all respects who has many good reasons for quitting life.
\end{sparagraph}

\vspace{0.5em}
\begin{sparagraph}[title=39]
  He is no friend who is continually asking for help, nor he who never associates help with friendship.
  For the former barters kindly feeling for a practical return and the latter destroys the hope of good in the future.
\end{sparagraph}

\vspace{0.5em}
\begin{sparagraph}[title=40]
  The man who says that all things come to pass by necessity cannot criticize one who denies that all things come to pass by necessity: for he admits that this too happens of necessity.
\end{sparagraph}

\vspace{0.5em}
\begin{sparagraph}[title=41]
  We must laugh and philosophize at the same time and do our household duties and employ our other faculties, and never cease proclaiming the sayings of the true philosophy.
\end{sparagraph}

\vspace{0.5em}
\begin{sparagraph}[title=42]
  The greatest blessing is created and enjoyed at the same moment.
\end{sparagraph}

\vspace{0.5em}
\begin{sparagraph}[title=43]
  The love of money, if unjustly gained, is impious, and, if justly, shameful; for it is unseemly to be merely parsimonious even with justice on one's side.
\end{sparagraph}

\vspace{0.5em}
\begin{sparagraph}[title=44]
  The wise man when he has accommodated himself to straits knows better how to give than to receive: so great is the treasure of self-sufficiency which he has discovered.
\end{sparagraph}

\vspace{0.5em}
\begin{sparagraph}[title=45]
  The study of nature does not make men productive of boasting or bragging nor apt to display that culture which is the object of rivalry with the many, but high-spirited and self-sufficient, taking pride in the good things of their own minds and not of their circumstances.
\end{sparagraph}

\vspace{0.5em}
\begin{sparagraph}[title=46]
  Our bad habits, like evil men who have long done us great harm, let us utterly drive from us.
\end{sparagraph}

\vspace{0.5em}
\begin{sparagraph}[title=47]
  I have anticipated thee, Fortune, and entrenched myself against all thy secret attacks.
  And we will not give ourselves up as captives to thee or to any other circumstance; but when it is time for us to go, spitting contempt on life and on those who here vainly cling to it, we will leave life crying aloud in a glorious triumph-song that we have lived well.
\end{sparagraph}

\vspace{0.5em}
\begin{sparagraph}[title=48]
  We must try to make the end of the journey better than the beginning, as long as we are journeying, but when we come to the end, we must be happy and content.
\end{sparagraph}

\vspace{0.5em}
\begin{sparagraph}[title=51]
  You tell me that the stimulus of the flesh makes you too prone to the pleasures of love.
  Provided that you do not break the laws or good customs and do not distress any of your neighbours or do harm to your body or squander your pittance, you may indulge your inclination as you please.
  Yet it is impossible not to come up against one or other of these barriers: for the pleasures of love never profited a man and he is lucky if they do him no harm.
\end{sparagraph}

\vspace{0.5em}
\begin{sparagraph}[title=52]
  Friendship goes dancing round the world proclaiming to us all to awake to the praises of a happy life.
\end{sparagraph}

\vspace{0.5em}
\begin{sparagraph}[title=53]
  We must envy no one: for the good do not deserve envy and the bad, the more they prosper, the more they injure themselves.
\end{sparagraph}

\vspace{0.5em}
\begin{sparagraph}[title=54]
  We must not pretend to study philosophy, but study it in reality: for it is not the appearance of health that we need, but real health.
\end{sparagraph}

\vspace{0.5em}
\begin{sparagraph}[title=55]
  We must heal our misfortunes by the grateful recollection of what has been and by the recognition that it is impossible to make undone what has been done.
\end{sparagraph}

\vspace{0.5em}
\begin{sparagraph}[title=56--57]
  The wise man is not more pained when being tortured textins{himself, than when seeing} his friend \textins{tortured}: textins{but if his friend does him wrong}, his whole life will be confounded by distrust and completely upset.
\end{sparagraph}

\vspace{0.5em}
\begin{sparagraph}[title=58]
  We must release ourselves from the prison of affairs and politics.
\end{sparagraph}

\vspace{0.5em}
\begin{sparagraph}[title=59]
  It is not the stomach that is insatiable, as is generally said, but the false opinion that the stomach needs an unlimited amount to fill it.
\end{sparagraph}

\vspace{0.5em}
\begin{sparagraph}[title=60]
  Every man passes out of life as though he had just been born.
\end{sparagraph}

\vspace{0.5em}
\begin{sparagraph}[title=61]
  Most beautiful too is the sight of those near and dear to us, when our original kinship makes us of one mind; for such sight is a great incitement to this end.
\end{sparagraph}

\vspace{0.5em}
\begin{sparagraph}[title=62]
  Now if parents are justly angry with their children, it is certainly useless to fight against it and not to ask for pardon; but if their anger is unjust and irrational, it is quite ridiculous to add fuel to their irrational passion by nursing one’s own indignation, and not to attempt to turn aside their wrath in other ways by gentleness.
\end{sparagraph}

\vspace{0.5em}
\begin{sparagraph}[title=63]
  Frugality too has a limit, and the man who disregards it is in like case with him who errs through excess.
\end{sparagraph}

\vspace{0.5em}
\begin{sparagraph}[title=64]
  Praise from others must come unasked: we must concern ourselves with the healing of our own lives.
\end{sparagraph}

\vspace{0.5em}
\begin{sparagraph}[title=65]
  It is vain to ask of the gods what a man is capable of supplying for himself.
\end{sparagraph}

\vspace{0.5em}
\begin{sparagraph}[title=66]
  Let us show our feeling for our lost friends not by lamentation but by meditation.
\end{sparagraph}

\vspace{0.5em}
\begin{sparagraph}[title=67]
  A free life cannot acquire many possessions, because this is not easy to do without servility to mobs or monarchs, yet it possesses all things in unfailing abundance; and if by chance it obtains many possessions, it is easy to distribute them so as to win the gratitude of neighbours.
\end{sparagraph}

\vspace{0.5em}
\begin{sparagraph}[title=68]
  Nothing is suflicient for him to whom what is suflicient seems little.
\end{sparagraph}

\vspace{0.5em}
\begin{sparagraph}[title=69]
  The ungrateful greed of the soul makes the creature everlastingly desire varieties of dainty food.
\end{sparagraph}

\vspace{0.5em}
\begin{sparagraph}[title=70]
  Let nothing be done in your life, which will cause you fear if it becomes known to your neighbour.
\end{sparagraph}

\vspace{0.5em}
\begin{sparagraph}[title=71]
  Every desire must be confronted with this question: what will happen to me, if the object of my desire is accomplished and what if it is not?
\end{sparagraph}

\vspace{0.5em}
\begin{sparagraph}[title=73]
  The occurrence of certain bodily pains assists us in guarding against others like them.
\end{sparagraph}

\vspace{0.5em}
\begin{sparagraph}[title=74]
  In a philosophical discussion he who is worsted gains more in proportion as he learns more.
\end{sparagraph}

\vspace{0.5em}
\begin{sparagraph}[title=75]
  Ungrateful towards the blessings of the past is the saying, Wait till the end of a long life.
\end{sparagraph}

\vspace{0.5em}
\begin{sparagraph}[title=76]
  You are in your old age just such as I urge you to be, and you have seen the difference between studying philosophy for oneself and proclaiming it to \sr{ancient Greece}{Greece} at large: I rejoice with you.
\end{sparagraph}

\vspace{0.5em}
\begin{sparagraph}[title=77]
  The greatest fruit of self-sufliciency is freedom.
\end{sparagraph}

\vspace{0.5em}
\begin{sparagraph}[title=78]
  The noble soul occupies itself with wisdom and friendship: of these the one is a mortal good, the other immortal.
\end{sparagraph}

\vspace{0.5em}
\begin{sparagraph}[title=79]
  The man who is serene causes no disturbance to himself or to another.
\end{sparagraph}

\vspace{0.5em}
\begin{sparagraph}[title=80]
  The first measure of security is to watch over one’s youth and to guard against what makes havoc of all by means of pestering desires.
\end{sparagraph}

\vspace{0.5em}
\begin{sparagraph}[title=81]
  The disturbance of the soul cannot be ended nor true joy created either by the possession of the greatest wealth or by honour and respect in the eyes of the mob or by anything else that is associated with causes of unlimited desire.
\end{sparagraph}
\end{document}
