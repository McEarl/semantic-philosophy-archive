\documentclass{stex}
\libinput{preamble}
\begin{document}
\usemodule{glossary/persons?epicurus}
\usemodule{glossary/persons?pythocles}
\usemodule{glossary/persons?herodotus}
\usemodule{glossary/persons?cleon}
\usemodule{glossary/writings/epicurus?letter-to-herodotus}
\usemodule{glossary/writings/epicurus?on-nature}

\noindent \sn{Epicurus} to \sn{Pythocles}, greeting.

In your letter to me, of which \sn{Cleon} was the bearer, you continue to show me affection which I have merited by my devotion to you, and you try, not without success, to recall the considerations which make for a happy life.
To aid your memory you ask me for a clear and concise statement respecting celestial phenomena; for what we have written on this subject elsewhere is, you tell me, hard to remember, although you have my books constantly with you.
I was glad to receive your request and am full of pleasant expectations.
We will then complete our writing and grant all you ask.
Many others besides you will find these reasonings useful, and especially those who have but recently made acquaintance with the true story of nature and those who are attached to pursuits which go deeper than any part of ordinary education.
So you will do well to take and learn them and get them up quickly along with the short epitome in my \sr{Letter to Herodotus}{letter to \sn{Herodotus}}.

In the first place, remember that, like everything else, knowledge of celestial phenomena, whether taken along with other things or in isolation, has no other end in view than peace of mind and firm conviction.
We do not seek to wrest by force what is impossible, nor to understand all matters equally well, nor make our treatment always as clear as when we discuss human life or explain the principles of physics in general – for instance, that the whole of being consists of bodies and intangible nature, or that the ultimate elements of things are indivisible, or any other proposition which admits only one explanation of the phenomena to be possible.
But this is not the case with celestial phenomena: these at any rate admit of manifold causes for their occurrence and manifold accounts, none of them contradictory of sensation, of their nature.

For in the study of nature we must not conform to empty assumptions and arbitrary laws, but follow the promptings of the facts; for our life has no need now of unreason and false opinion; our one need is untroubled existence.
All things go on uninterruptedly, if all be explained by the method of plurality of causes in conformity with the facts, so soon as we duly understand what may be plausibly alleged respecting them.
But when we pick and choose among them, rejecting one equally consistent with the phenomena, we clearly fall away from the study of nature altogether and tumble into myth.
Some phenomena within our experience afford evidence by which we may interpret what goes on in the heavens.
We see how the former really take place, but not how the celestial phenomena take place, for their occurrence may possibly be due to a variety of causes.
However, we must observe each fact as presented, and further separate from it all the facts presented along with it, the occurrence of which from various causes is not contradicted by facts within our experience.

A world is a circumscribed portion of the universe, which contains stars and earth and all other visible things, cut off from the infinite, and terminating in an exterior which may either revolve or be at rest, and be round or triangular or of any other shape whatever.
All these alternatives are possible: they are contradicted by none of the facts in this world, in which an extremity can nowhere be discerned.

That there is an infinite number of such worlds can be perceived, and that such a world may arise in a world or in one of the \foreignlanguage{latin}{intermundia} (by which term we mean the spaces between worlds) in a tolerably empty space and not, as some maintain, in a vast space perfectly clear and void.
It arises when certain suitable seeds rush in from a single world or \foreignlanguage{latin}{intermundium}, or from several, and undergo gradual additions or articulations or changes of place, it may be, and waterings from appropriate sources, until they are matured and firmly settled in so far as the foundations laid can receive them.
For it is not enough that there should be an aggregation or a vortex in the empty space in which a world may arise, as the necessitarians hold, and may grow until it collide with another, as one of the so-called physicists says.
For this is in conflict with facts.

The sun and moon and the stars generally were not of independent origin and later absorbed within our world; but they at once began to take form and grow by the accretions and whirling motions of certain substances of finest texture, of the nature either of wind or fire, or of both; for thus sense itself suggests.

The size of the sun and the remaining stars relatively to us is just as great as it appears.
But in itself and actually it may be a little larger or a little smaller, or precisely as great as it is seen to be.
For so too fires of which we have experience are seen by sense when we see them at a distance.
And every objection brought against this part of the theory will easily be met by anyone who attends to plain facts, as I show in my work \sn{On Nature}.
And the rising and setting of the sun, moon, and stars may be due to kindling and quenching, provided that the circumstances are such as to produce this result in each of the two regions, east and west: for no fact testifies against this.
Or the result might be produced by their coming forward above the earth and again by its intervention to hide them: for no fact testifies against this either.
And their motions may be due to the rotation of the whole heaven, or the heaven may be at rest and they alone rotate according to some necessary impulse to rise, implanted at first when the world was made \textelp{} and this through excessive heat, due to a certain extension of the fire which always encroaches upon that which is near it.

The turnings of the sun and moon in their course may be due to the obliquity of the heaven, whereby it is forced back at these times.
Again, they may equally be due to the contrary pressure of the air or, it may be, to the fact that either the fuel from time to time necessary has been consumed in the vicinity or there is a dearth of it.
Or even because such a whirling motion was from the first inherent in these stars so that they move in a sort of spiral.
For all such explanations and the like do not conflict with any clear evidence, if only in such details we hold fast to what is possible, and can bring each of these explanations into accord with the facts, unmoved by the servile artifices of the astronomers.

The waning of the moon and again her waxing might be due to the rotation of the moon's body, and equally well to configurations which the air assumes; further, it may be due to the interposition of certain bodies.
In short, it may happen in any of the ways in which the facts within our experience suggest such an appearance to be explicable.
But one must not be so much in love with the explanation by a single way as wrongly to reject all the others from ignorance of what can, and what cannot, be within human knowledge, and consequent longing to discover the indiscoverable.
Further, the moon may possibly shine by her own light, just as possibly she may derive her light from the sun; for in our own experience we see many things which shine by their own light and many also which shine by borrowed light.
And none of the celestial phenomena stand in the way, if only we always keep in mind the method of plural explanation and the several consistent assumptions and causes, instead of dwelling on what is inconsistent and giving it a false importance so as always to fall back in one way or another upon the single explanation.
The appearance of the face in the moon may equally well arise from interchange of parts, or from interposition of something, or in any other of the ways which might be seen to accord with the facts.
For in all the celestial phenomena such a line of research is not to be abandoned; for, if you fight against clear evidence, you never can enjoy genuine peace of mind.

An eclipse of the sun or moon may be due to the extinction of their light, just as within our own experience this is observed to happen; and again by interposition of something else – whether it be the earth or some other invisible body like it.
And thus we must take in conjunction the explanations which agree with one another, and remember that the concurrence of more than one at the same time may not impossibly happen.

And further, let the regularity of their orbits be explained in the same way as certain ordinary incidents within our own experience; the divine nature must not on any account be adduced to explain this, but must be kept free from the task and in perfect bliss.
Unless this be done, the whole study of celestial phenomena will be in vain, as indeed it has proved to be with some who did not lay hold of a possible method, but fell into the folly of supposing that these events happen in one single way only and of rejecting all the others which are possible, suffering themselves to be carried into the realm of the unintelligible, and being unable to take a comprehensive view of the facts which must be taken as clues to the rest.

The variations in the length of nights and days may be due to the swiftness and again to the slowness of the sun's motion in the sky, owing to the variations in the length of spaces traversed and to his accomplishing some distances more swiftly or more slowly, as happens sometimes within our own experience; and with these facts our explanation of celestial phenomena must agree; whereas those who adopt only one explanation are in conflict with the facts and are utterly mistaken as to the way in which man can attain knowledge.

The signs in the sky which betoken the weather may be due to mere coincidence of the seasons, as is the case with signs from animals seen on earth, or they may be caused by changes and alterations in the air.
For neither the one explanation nor the other is in conflict with facts, and it is not easy to see in which cases the effect is due to one cause or to the other.

Clouds may form and gather either because the air is condensed under the pressure of winds, or because atoms which hold together and are suitable to produce this result become mutually entangled, or because currents collect from the earth and the waters; and there are several other ways in which it is not impossible for the aggregations of such bodies into clouds to be brought about.
And that being so, rain may be produced from them sometimes by their compression, sometimes by their transformation; or again may be caused by exhalations of moisture rising from suitable places through the air, while a more violent inundation is due to certain accumulations suitable for such discharge.
Thunder may be due to the rolling of wind in the hollow parts of the clouds, as it is sometimes imprisoned in vessels which we use; or to the roaring of fire in them when blown by a wind, or to the rending and disruption of clouds, or to the friction and splitting up of clouds when they have become as firm as ice.
As in the whole survey, so in this particular point, the facts invite us to give a plurality of explanations.
Lightnings too happen in a variety of ways.
For when the clouds rub against each other and collide, that collocation of atoms which is the cause of fire generates lightning; or it may be due to the flashing forth from the clouds, by reason of winds, of particles capable of producing this brightness; or else it is squeezed out of the clouds when they have been condensed either by their own action or by that of the winds; or again, the light diffused from the stars may be enclosed in the clouds, then driven about by their motion and by that of the winds, and finally make its escape from the clouds; or light of the finest texture may be filtered through the clouds (whereby the clouds may be set on fire and thunder produced), and the motion of this light may make lightning; or it may arise from the combustion of wind brought about by the violence of its motion and the intensity of its compression;
or, when the clouds are rent asunder by winds, and the atoms which generate fire are expelled, these likewise cause lightning to appear.
And it may easily be seen that its occurrence is possible in many other ways, so long as we hold fast to facts and take a general view of what is analogous to them.
Lightning precedes thunder, when the clouds are constituted as mentioned above and the configuration which produces lightning is expelled at the moment when the wind falls upon the cloud, and the wind being rolled up afterwards produces the roar of thunder; or, if both are simultaneous, the lightning moves with a greater velocity towards us and the thunder lags behind, exactly as when persons who are striking blows are observed from a distance.
A thunderbolt is caused when winds are repeatedly collected, imprisoned, and violently ignited; or when a part is torn asunder and is more violently expelled downwards, the rending being due to the fact that the compression of the clouds has made the neighbouring parts more dense; or again it may be due like thunder merely to the expulsion of the imprisoned fire, when this has accumulated and been more violently inflated with wind and has torn the cloud, being unable to withdraw to the adjacent parts because it is continually more and more closely compressed.
And there are several other ways in which thunderbolts may possibly be produced.
Exclusion of myth is the sole condition necessary; and it will be excluded, if one properly attends to the facts and hence draws inferences to interpret what is obscure.

Fiery whirlwinds are due to the descent of a cloud forced downwards like a pillar by the wind in full force and carried by a gale round and round, while at the same time the outside wind gives the cloud a lateral thrust; or it may be due to a change of the wind which veers to all points of the compass as a current of air from above helps to force it to move; or it may be that a strong eddy of winds has been started and is unable to burst through laterally because the air around is closely condensed.
And when they descend upon land, they cause what are called tornadoes, in accordance with the various ways in which they are produced through the force of the wind; and when let down upon the sea, they cause waterspouts.

Earthquakes may be due to the imprisonment of wind underground, and to its being interspersed with small masses of earth and then set in continuous motion, thus causing the earth to tremble.
And the earth either takes in this wind from without or from the falling in of foundations, when undermined, into subterranean caverns, thus raising a wind in the imprisoned air.
Or they may be due to the propagation of movement arising from the fall of many foundations and to its being again checked when it encounters the more solid resistance of earth.
And there are many other causes to which these oscillations of the earth may be due.

Winds arise from time to time when foreign matter continually and gradually finds its way into the air; also through the gathering of great store of water.
The rest of the winds arise when a few of them fall into the many hollows and they are thus divided and multiplied.

Hail is caused by the firmer congelation and complete transformation, and subsequent distribution into drops, of certain particles resembling wind: also by the slighter congelation of certain particles of moisture and the vicinity of certain particles of wind which at one and the same time forces them together and makes them burst, so that they become frozen in parts and in the whole mass.
The round shape of hailstones is not impossibly due to the extremities on all sides being melted and to the fact that, as explained, particles either of moisture or of wind surround them evenly on all sides and in every quarter, when they freeze.

Snow may be formed when a fine rain issues from the clouds because the pores are symmetrical and because of the continuous and violent pressure of the winds upon clouds which are suitable; and then this rain has been frozen on its way because of some violent change to coldness in the regions below the clouds.
Or again, by congelation in clouds which have uniform density a fall of snow might occur through the clouds which contain moisture being densely packed in close proximity to each other; and these clouds produce a sort of compression and cause hail, and this happens mostly in spring.
And when frozen clouds rub against each other, this accumulation of snow might be thrown off.
And there are other ways in which snow might be formed.

Dew is formed when such particles as are capable of producing this sort of moisture meet each other from the air: again by their rising from moist and damp places, the sort of place where dew is chiefly formed, and their subsequent coalescence, so as to create moisture and fall downwards, just as in several cases something similar is observed to take place under our eyes.

And the formation of hoar-frost is not different from that of dew, certain particles of such a nature becoming in some such way congealed owing to a certain condition of cold air.

Ice is formed by the expulsion from the water of the circular, and the compression of the scalene and acute-angled atoms contained in it; further by the accretion of such atoms from without, which being driven together cause the water to solidify after the expulsion of a certain number of round atoms.

The rainbow arises when the sun shines upon humid air; or again by a certain peculiar blending of light with air, which will cause either all the distinctive qualities of these colours or else some of them belonging to a single kind, and from the reflection of this light the air all around will be coloured as we see it to be, as the sun shines upon its parts.
The circular shape which it assumes is due to the fact that the distance of every point is perceived by our sight to be equal; or it may be because, the atoms in the air or in the clouds and deriving from the sun having been thus united, the aggregate of them presents a sort of roundness.

A halo round the moon arises because the air on all sides extends to the moon; or because it equably raises upwards the currents from the moon so high as to impress a circle upon the cloudy mass and not to separate it altogether; or because it raises the air which immediately surrounds the moon symmetrically from all sides up to a circumference round her and there forms a thick ring.
And this happens at certain parts either because a current has forced its way in from without or because the heat has gained possession of certain passages in order to effect this.

Comets arise either because fire is nourished in certain places at certain intervals in the heavens, if circumstances are favourable; or because at times the heaven has a particular motion above us so that such stars appear; or because the stars themselves are set in motion under certain conditions and come to our neighbourhood and show themselves.
And their disappearance is due to the causes which are the opposite of these.
Certain stars may revolve without setting not only for the reason alleged by some, because this is the part of the world round which, itself unmoved, the rest revolves, but it may also be because a circular eddy of air surrounds this part, which prevents them from travelling out of sight like other stars; or because there is a dearth of necessary fuel farther on, while there is abundance in that part where they are seen to be.
Moreover there are several other ways in which this might be brought about, as may be seen by anyone capable of reasoning in accordance with the facts.
The wanderings of certain stars, if such wandering is their actual motion, and the regular movement of certain other stars, may be accounted for by saying that they originally moved in a circle and were constrained, some of them to be whirled round with the same uniform rotation and others with a whirling motion which varied; but it may also be that according to the diversity of the regions traversed in some places there are uniform tracts of air, forcing them forward in one direction and burning uniformly, in others these tracts present such irregularities as cause the motions observed.
To assign a single cause for these effects when the facts suggest several causes is madness and a strange inconsistency; yet it is done by adherents of rash astronomy, who assign meaningless causes for the stars whenever they persist in saddling the divinity with burdensome tasks.
That certain stars are seen to be left behind by others may be because they travel more slowly, though they go the same round as the others; or it may be that they are drawn back by the same whirling motion and move in the opposite direction; or again it may be that some travel over a larger and others over a smaller space in making the same revolution.
But to lay down as assured a single explanation of these phenomena is worthy of those who seek to dazzle the multitude with marvels.

Falling stars, as they are called, may in some cases be due to the mutual friction of the stars themselves, in other cases to the expulsion of certain parts when that mixture of fire and air takes place which was mentioned when we were discussing lightning; or it may be due to the meeting of atoms capable of generating fire, which accord so well as to produce this result, and their subsequent motion wherever the impulse which brought them together at first leads them; or it may be that wind collects in certain dense mist-like masses and, since it is imprisoned, ignites and then bursts forth upon whatever is round about it, and is carried to that place to which its motion impels it.
And there are other ways in which this can be brought about without recourse to myths.

The fact that the weather is sometimes foretold from the behaviour of certain animals is a mere coincidence in time.
For the animals offer no necessary reason why a storm should be produced; and no divine being sits observing when these animals go out and afterwards fulfilling the signs which they have given.
For such folly as this would not possess the most ordinary being if ever so little enlightened, much less one who enjoys perfect felicity.

All this, \sn{Pythocles}, you should keep in mind; for then you will escape a long way from myth, and you will be able to view in their connexion the instances which are similar to these.
But above all give yourself up to the study of first principles and of infinity and of kindred subjects, and further of the standards and of the feelings and of the end for which we choose between them.
For to study these subjects together will easily enable you to understand the causes of the particular phenomena.
And those who have not fully accepted this, in proportion as they have not done so, will be ill acquainted with these very subjects, nor have they secured the end for which they ought to be studied.
\end{document}
