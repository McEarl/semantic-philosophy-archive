\documentclass{stex}
\libinput{preamble}
\begin{document}
\usemodule{glossary/persons?epicurus}
\usemodule{glossary/persons?herodotus}
\usemodule{glossary/abbreviations?id-est}
\usemodule{glossary/abbreviations?videlicet}

\noindent \sn{Epicurus} to \sn{Herodotus}, greeting.

For those who are unable to study carefully all my physical writings or to go into the longer treatises at all, I have myself prepared an epitome of the whole system, \sn{Herodotus}, to preserve in the memory enough of the principal doctrines, to the end that on every occasion they may be able to aid themselves on the most important points, so far as they take up the study of Physics.
Those who have made some advance in the survey of the entire system ought to fix in their minds under the principal headings an elementary outline of the whole treatment of the subject.
For a comprehensive view is often required, the details but seldom.

To the former, then – the main heads – we must continually return, and must memorize them so far as to get a valid conception of the facts, as well as the means of discovering all the details exactly when once the general outlines are rightly understood and remembered; since it is the privilege of the mature student to make a ready use of his conceptions by referring every one of them to elementary facts and simple terms.
For it is impossible to gather up the results of continuous diligent study of the entirety of things, unless we can embrace in short formulas and hold in mind all that might have been accurately expressed even to the minutest detail.

Hence, since such a course is of service to all who take up natural science, I, who devote to the subject my continuous energy and reap the calm enjoyment of a life like this, have prepared for you just such an epitome and manual of the doctrines as a whole.

In the first place, \sn{Herodotus}, you must understand what it is that words denote, in order that by reference to this we may be in a position to test opinions, inquiries, or problems, so that our proofs may not run on untested \foreignlanguage{latin}{ad infinitum}, nor the terms we use be empty of meaning.
For the primary signification of every term employed must be clearly seen, and ought to need no proving; this being necessary, if we are to have something to which the point at issue or the problem or the opinion before us can be referred.

Next, we must by all means stick to our sensations, that is, simply to the present impressions whether of the mind or of any criterion whatever, and similarly to our actual feelings, in order that we may have the means of determining that which needs confirmation and that which is obscure.

When this is clearly understood, it is time to consider generally things which are obscure.
To begin with, nothing comes into being out of what is non-existent.
For in that case anything would have arisen out of anything, standing as it would in no need of its proper germs.
And if that which disappears had been destroyed and become non-existent, everything would have perished, that into which the things were dissolved being non-existent.
Moreover, the sum total of things was always such as it is now, and such it will ever remain.
For there is nothing into which it can change.
For outside the sum of things there is nothing which could enter into it and bring about the change.

Further, the whole of being consists of bodies and space.
For the existence of bodies is everywhere attested by sense itself, and it is upon sensation that reason must rely when it attempts to infer the unknown from the known.
And if there were no space (which we call also void and place and intangible nature), bodies would have nothing in which to be and through which to move, as they are plainly seen to move.
Beyond bodies and space there is nothing which by mental apprehension or on its analogy we can conceive to exist.
When we speak of bodies and space, both are regarded as wholes or separate things, not as the properties or accidents of separate things.

Again, of bodies some are composite, others the elements of which these composite bodies are made.
These elements are indivisible and unchangeable, and necessarily so, if things are not all to be destroyed and pass into non-existence, but are to be strong enough to endure when the composite bodies are broken up, because they possess a solid nature and are incapable of being anywhere or anyhow dissolved.
It follows that the first beginnings must be indivisible, corporeal entities.

Again, the sum of things is infinite.
For what is finite has an extremity, and the extremity of anything is discerned only by comparison with something else.
(Now the sum of things is not discerned by comparison with anything else: hence, since it has no extremity, it has no limit; and, since it has no limit, it must be unlimited or infinite.

Moreover, the sum of things is unlimited both by reason of the multitude of the atoms and the extent of the void.
For if the void were infinite and bodies finite, the bodies would not have stayed anywhere but would have been dispersed in their course through the infinite void, not having any supports or counter-checks to send them back on their upward rebound.
Again, if the void were finite, the infinity of bodies would not have anywhere to be.

Furthermore, the atoms, which have no void in them – out of which composite bodies arise and into which they are dissolved – vary indefinitely in their shapes; for so many varieties of things as we see could never have arisen out of a recurrence of a definite number of the same shapes.
The like atoms of each shape are absolutely infinite; but the variety of shapes, though indefinitely large, is not absolutely infinite.

The atoms are in continual motion through all eternity.
Some of them rebound to a considerable distance from each other, while others merely oscillate in one place when they chance to have got entangled or to be enclosed by a mass of other atoms shaped for entangling.

This is because each atom is separated from the rest by void, which is incapable of offering any resistance to the rebound; while it is the solidity of the atom which makes it rebound after a collision, however short the distance to which it rebounds, when it finds itself imprisoned in a mass of entangling atoms.
Of all this there is no beginning, since both atoms and void exist from everlasting.

The repetition at such length of all that we are now recalling to mind furnishes an adequate outline for our conception of the nature of things.

Moreover, there is an infinite number of worlds, some like this world, others unlike it.
For the atoms being infinite in number, as has just been proved, are borne ever further in their course.
For the atoms out of which a world might arise, or by which a world might be formed, have not all been expended on one world or a finite number of worlds, whether like or unlike this one.
Hence there will be nothing to hinder an infinity of worlds.

Again, there are outlines or films, which are of the same shape as solid bodies, but of a thinness far exceeding that of any object that we see.
For it is not impossible that there should be found in the surrounding air combinations of this kind, materials adapted for expressing the hollowness and thinness of surfaces, and effluxes preserving the same relative position and motion which they had in the solid objects from which they come.
To these films we give the name of \enquote{images} or \enquote{idols}.
Furthermore, so long as nothing comes in the way to offer resistance, motion through the void accomplishes any imaginable distance in an inconceivably short time.
For resistance encountered is the equivalent of slowness, its absence the equivalent of speed.

Not that, if we consider the minute times perceptible by reason alone, the moving body itself arrives at more than one place simultaneously (for this too is inconceivable), although in time perceptible to sense it does arrive simultaneously, however different the point of departure from that conceived by us.
For if it changed its direction, that would be equivalent to its meeting with resistance, even if up to that point we allow nothing to impede the rate of its flight.
This is an elementary fact which in itself is well worth bearing in mind.
In the next place the exceeding thinness of the images is contradicted by none of the facts under our observation.
Hence also their velocities are enormous, since they always find a void passage to fit them.
Besides, their incessant effluence meets with no resistance, or very little, although many atoms, not to say an unlimited number, do at once encounter resistance.

Besides this, remember that the production of the images is as quick as thought.
For particles are continually streaming off from the surface of bodies, though no diminution of the bodies is observed, because other particles take their place.
And those given off for a long time retain the position and arrangement which their atoms had when they formed part of the solid bodies, although occasionally they are thrown into confusion.
Sometimes such films are formed very rapidly in the air, because they need not have any solid content; and there are other modes in which they may be formed.
For there is nothing in all this which is contradicted by sensation, if we in some sort look at the clear evidence of sense, to which we should also refer the continuity of particles in the objects external to ourselves.

We must also consider that it is by the entrance of something coming from external objects that we see their shapes and think of them.
For external things would not stamp on us their own nature of colour and form through the medium of the air which is between them and us, or by means of rays of light or currents of any sort going from us to them, so well as by the entrance into our eyes or minds, to whichever their size is suitable, of certain films coming from the things themselves, these films or outlines being of the same colour and shape as the external things themselves.
They move with rapid motion; and this again explains why they present the appearance of the single continuous object, and retain the mutual interconnexion which they had in the object, when they impinge upon the sense, such impact being due to the oscillation of the atoms in the interior of the solid object from which they come.
And whatever presentation we derive by direct contact, whether it be with the mind or with the sense-organs, be it shape that is presented or other properties, this shape as presented is the shape of the solid thing, and it is due either to a close coherence of the image as a whole or to a mere remnant of its parts.
Falsehood and error always depend upon the intrusion of opinion (when a fact awaits) confirmation or the absence of contradiction, which fact is afterwards frequently not confirmed (or even contradicted).

For the presentations which, e.g., are received in a picture or arise in dreams, or from any other form of apprehension by the mind or by the other criteria of truth, would never have resembled what we call the real and true things, had it not been for certain actual things of the kind with which we come in contact.
Error would not have occurred, if we had not experienced some other movement in ourselves, conjoined with, but distinct from, the perception of what is presented.
And from this movement, if it be not confirmed or be contradicted, falsehood results; while, if it be confirmed or not contradicted, truth results.

And to this view we must closely adhere, if we are not to repudiate the criteria founded on the clear evidence of sense, nor again to throw all these things into confusion by maintaining falsehood as if it were truth.

Again, hearing takes place when a current passes from the object, whether person or thing, which emits voice or sound or noise, or produces the sensation of hearing in any way whatever.
This current is broken up into homogeneous particles, which at the same time preserve a certain mutual connexion and a distinctive unity extending to the object which emitted them, and thus, for the most part, cause the perception in that case or, if not, merely indicate the presence of the external object.
For without the transmission from the object of a certain interconnexion of the parts no such sensation could arise.
Therefore we must not suppose that the air itself is moulded into shape by the voice emitted or something similar; for it is very far from being the case that the air is acted upon by it in this way.
The blow which is struck in us when we utter a sound causes such a displacement of the particles as serves to produce a current resembling breath, and this displacement gives rise to the sensation of hearing.

Again, we must believe that smelling, like hearing, would produce no sensation, were there not particles conveyed from the object which are of the proper sort for exciting the organ of smelling, some of one sort, some of another, some exciting it confusedly and strangely, others quietly and agreeably.

Moreover, we must hold that the atoms in fact possess none of the qualities belonging to things which come under our observation, except shape, weight, and size, and the properties necessarily conjoined with shape.
For every quality changes, but the atoms do not change, since, when the composite bodies are dissolved, there must needs be a permanent something, solid and indissoluble, left behind, which makes change possible: not changes into or from the non-existent, but often through differences of arrangement, and sometimes through additions and subtractions of the atoms.
Hence these somethings capable of being diversely arranged must be indestructible, exempt from change, but possessed each of its own distinctive mass and configuration.
This must remain.

For in the case of changes of configuration within our experience the figure is supposed to be inherent when other qualities are stripped off, but the qualities are not supposed, like the shape which is left behind, to inhere in the subject of change, but to vanish altogether from the body.
Thus, then, what is left behind is sufficient to account for the differences in composite bodies, since something at least must necessarily be left remaining and be immune from annihilation.

Again, you should not suppose that the atoms have any and every size, lest you be contradicted by facts; but differences of size must be admitted; for this addition renders the facts of feeling and sensation easier of explanation.
But to attribute any and every magnitude to the atoms does not help to explain the differences of quality in things; moreover, in that case atoms large enough to be seen ought to have reached us, which is never observed to occur; nor can we conceive how its occurrence should be possible, \idEst{} that an atom should become visible.

Besides, you must not suppose that there are parts unlimited in number, be they ever so small, in any finite body.
Hence not only must we reject as impossible subdivision \foreignlanguage{latin}{ad infinitum} into smaller and smaller parts, lest we make all things too weak and, in our conceptions of the aggregates, be driven to pulverize the things that exist, \idEst{} the atoms, and annihilate them; but in dealing with finite things we must also reject as impossible the progression \foreignlanguage{latin}{ad infinitum} by less and less increments.

For when once we have said that an infinite number of particles, however small, are contained in anything, it is not possible to conceive how it could any longer be limited or finite in size.
For clearly our infinite number of particles must have some size; and then, of whatever size they were, the aggregate they made would be infinite.
And, in the next place, since what is finite has an extremity which is distinguishable, even if it is not by itself observable, it is not possible to avoid thinking of another such extremity next to this.
Nor can we help thinking that in this way, by proceeding forward from one to the next in order, it is possible by such a progression to arrive in thought at infinity.

We must consider the minimum perceptible by sense as not corresponding to that which is capable of being traversed, \idEst{} is extended, nor again as utterly unlike it, but as having something in common with the things capable of being traversed, though it is without distinction of parts.
But when from the illusion created by this common property we think we shall distinguish something in the minimum, one part on one side and another part on the other side, it must be another minimum equal to the first which catches our eye.
In fact, we see these minima one after another, beginning with the first, and not as occupying the same space; nor do we see them touch one another's parts with their parts, but we see that by virtue of their own peculiar character (\idEst{} as being unit indivisibles) they afford a means of measuring magnitudes: there are more of them, if the magnitude measured is greater; fewer of them, if the magnitude measured is less.

We must recognize that this analogy also holds of the minimum in the atom; it is only in minuteness that it differs from that which is observed by sense, but it follows the same analogy.
On the analogy of things within our experience we have declared that the atom has magnitude; and this, small as it is, we have merely reproduced on a larger scale.
And further, the least and simplest things must be regarded as extremities of lengths, furnishing from themselves as units the means of measuring lengths, whether greater or less, the mental vision being employed, since direct observation is impossible.
For the community which exists between them and the unchangeable parts (\idEst{} the minimal parts of area or surface) is sufficient to justify the conclusion so far as this goes.
But it is not possible that these minima of the atom should group themselves together through the possession of motion.

Further, we must not assert \enquote{up} or \enquote{down} of that which is unlimited, as if there were a zenith or nadir.
As to the space overhead, however, if it be possible to draw a line to infinity from the point where we stand, we know that never will this space – or, for that matter, the space below the supposed standpoint if produced to infinity – appear to us to be at the same time \enquote{up} and \enquote{down} with reference to the same point; for this is inconceivable.
Hence it is possible to assume one direction of motion, which we conceive as extending upwards \foreignlanguage{latin}{ad infinitum}, and another downwards, even if it should happen ten thousand times that what moves from us to the spaces above our heads reaches the feet of those above us, or that which moves downwards from us the heads of those below us.
None the less is it true that the whole of the motion in the respective cases is conceived as extending in opposite directions \foreignlanguage{latin}{ad infinitum}.

When they are travelling through the void and meet with no resistance, the atoms must move with equal speed.
Neither will heavy atoms travel more quickly than small and light ones, so long as nothing meets them, nor will small atoms travel more quickly than large ones, provided they always find a passage suitable to their size, and provided also that they meet with no obstruction.
Nor will their upward or their lateral motion, which is due to collisions, nor again their downward motion, due to weight, affect their velocity.
As long as either motion obtains, it must continue, quick as the speed of thought, provided there is no obstruction, whether due to external collision or to the atoms' own weight counteracting the force of the blow.

Moreover, when we come to deal with composite bodies, one of them will travel faster than another, although their atoms have equal speed.
This is because the atoms in the aggregates are travelling in one direction during the shortest continuous time, albeit they move in different directions in times so short as to be appreciable only by the reason, but frequently collide until the continuity of their motion is appreciated by sense.
For the assumption that beyond the range of direct observation even the minute times conceivable by reason will present continuity of motion is not true in the case before us.
Our canon is that direct observation by sense and direct apprehension by the mind are alone invariably true.

Next, keeping in view our perceptions and feelings (for so shall we have the surest grounds for belief), we must recognize generally that the soul is a corporeal thing, composed of fine particles, dispersed all over the frame, most nearly resembling wind with an admixture of heat, in some respects like wind, in others like heat.
But, again, there is the third part which exceeds the other two in the fineness of its particles and thereby keeps in closer touch with the rest of the frame.
And this is shown by the mental faculties and feelings, by the ease with which the mind moves, and by thoughts, and by all those things the loss of which causes death.
Further, we must keep in mind that soul has the greatest share in causing sensation.
Still, it would not have had sensation, had it not been somehow confined within the rest of the frame.
But the rest of the frame, though it provides this indispensable condition for the soul, itself also has a share, derived from the soul, of the said quality; and yet does not possess all the qualities of soul.
Hence on the departure of the soul it loses sentience.
For it had not this power in itself; but something else, congenital with the body, supplied it to body: which other thing, through the potentiality actualized in it by means of motion, at once acquired for itself a quality of sentience, and, in virtue of the neighbourhood and interconnexion between them, imparted it (as I said) to the body also.

Hence, so long as the soul is in the body, it never loses sentience through the removal of some other part.
The containing sheath may be dislocated in whole or in part, and portions of the soul may thereby be lost; yet in spite of this the soul, if it manage to survive, will have sentience.
But the rest of the frame, whether the whole of it survives or only a part, no longer has sensation, when once those atoms have departed, which, however few in number, are required to constitute the nature of soul.
Moreover, when the whole frame is broken up, the soul is scattered and has no longer the same powers as before, nor the same motions; hence it does not possess sentience either.

For we cannot think of it as sentient, except it be in this composite whole and moving with these movements; nor can we so think of it when the sheaths which enclose and surround it are not the same as those in which the soul is now located and in which it performs these movements.

There is the further point to be considered, what the incorporeal can be, if, I mean, according to current usage the term is applied to what can be conceived as self-existent.
But it is impossible to conceive anything that is incorporeal as self-existent except empty space.
And empty space cannot itself either act or be acted upon, but simply allows body to move through it.
Hence those who call soul incorporeal speak foolishly.
For if it were so, it could neither act nor be acted upon.
But, as it is, both these properties, you see, plainly belong to soul.

If, then, we bring all these arguments concerning soul to the criterion of our feelings and perceptions, and if we keep in mind the proposition stated at the outset, we shall see that the subject has been adequately comprehended in outline: which will enable us to determine the details with accuracy and confidence.

Moreover, shapes and colours, magnitudes and weights, and in short all those qualities which are predicated of body, in so far as they are perpetual properties either of all bodies or of visible bodies, are knowable by sensation of these very properties: these, I say, must not be supposed to exist independently by themselves (for that is inconceivable), nor yet to be non-existent, nor to be some other and incorporeal entities cleaving to body, nor again to be parts of body.
We must consider the whole body in a general way to derive its permanent nature from all of them, though it is not, as it were, formed by grouping them together in the same way as when from the particles themselves a larger aggregate is made up, whether these particles be primary or any magnitudes whatsoever less than the particular whole.
All these qualities, I repeat, merely give the body its own permanent nature.
They all have their own characteristic modes of being perceived and distinguished, but always along with the whole body in which they inhere and never in separation from it; and it is in virtue of this complete conception of the body as a whole that it is so designated.

Again, qualities often attach to bodies without being permanent concomitants.
They are not to be classed among invisible entities nor are they incorporeal.
Hence, using the term \enquote{accidents} in the commonest sense, we say plainly that accidents have not the nature of the whole thing to which they belong, and to which, conceiving it as a whole, we give the name of body, nor that of the permanent properties without which body cannot be thought of.
And in virtue of certain peculiar modes of apprehension into which the complete body always enters, each of them can be called an accident.
But only as often as they are seen actually to belong to it, since such accidents are not perpetual concomitants.
There is no need to banish from reality this clear evidence that the accident has not the nature of that whole – by us called body – to which it belongs, nor of the permanent properties which accompany the whole.
Nor, on the other hand, must we suppose the accident to have independent existence (for this is just as inconceivable in the case of accidents as in that of the permanent properties); but, as is manifest, they should all be regarded as accidents, not as permanent concomitants, of bodies, nor yet as having the rank of independent existence.
Rather they are seen to be exactly as and what sensation itself makes them individually claim to be.

Here is another thing which we must consider carefully.
We must not investigate time as we do the other accidents which we investigate in a subject, namely, by referring them to the preconceptions envisaged in our minds; but we must take into account the plain fact itself, in virtue of which we speak of time as long or short, linking to it in intimate connexion this attribute of duration.
We need not adopt any fresh terms as preferable, but should employ the usual expressions about it.
Nor need we predicate anything else of time, as if this something else contained the same essence as is contained in the proper meaning of the word \enquote{time} (for this also is done by some).
We must chiefly reflect upon that to which we attach this peculiar character of time, and by which we measure it.
No further proof is required: we have only to reflect that we attach the attribute of time to days and nights and their parts, and likewise to feelings of pleasure and pain and to neutral states, to states of movement and states of rest, conceiving a peculiar accident of these to be this very characteristic which we express by the word \enquote{time}.

After the foregoing we have next to consider that the worlds and every finite aggregate which bears a strong resemblance to things we commonly see have arisen out of the infinite.
For all these, whether small or great, have been separated off from special conglomerations of atoms; and all things are again dissolved, some faster, some slower, some through the action of one set of causes, others through the action of another.

And further, we must not suppose that the worlds have necessarily one and the same shape.
For nobody can prove that in one sort of world there might not be contained, whereas in another sort of world there could not possibly be, the seeds out of which animals and plants arise and all the rest of the things we see.

Again, we must suppose that nature too has been taught and forced to learn many various lessons by the facts themselves, that reason subsequently develops what it has thus received and makes fresh discoveries, among some tribes more quickly, among others more slowly, the progress thus made being at certain times and seasons greater, at others less.

Hence even the names of things were not originally due to convention, but in the several tribes under the impulse of special feelings and special presentations of sense primitive man uttered special cries.
The air thus emitted was moulded by their individual feelings or sense-presentations, and differently according to the difference of the regions which the tribes inhabited.
Subsequently whole tribes adopted their own special names, in order that their communications might be less ambiguous to each other and more briefly expressed.
And as for things not visible, so far as those who were conscious of them tried to introduce any such notion, they put in circulation certain names for them, either sounds which they were instinctively compelled to utter or which they selected by reason on analogy according to the most general cause there can be for expressing oneself in such a way.

Nay more: we are bound to believe that in the sky revolutions, solstices, eclipses, risings and settings, and the like, take place without the ministration or command, either now or in the future, of any being who at the same time enjoys perfect bliss along with immortality.
For troubles and anxieties and feelings of anger and partiality do not accord with bliss, but always imply weakness and fear and dependence upon one's neighbours.
Nor, again, must we hold that things which are no more than globular masses of fire, being at the same time endowed with bliss, assume these motions at will.
Nay, in every term we use we must hold fast to all the majesty which attaches to such notions as bliss and immortality, lest the terms should generate opinions inconsistent with this majesty.
Otherwise such inconsistency will of itself suffice to produce the worst disturbance in our minds.
Hence, where we find phenomena invariably recurring, the invariableness of the recurrence must be ascribed to the original interception and conglomeration of atoms whereby the world was formed.

Further, we must hold that to arrive at accurate knowledge of the cause of things of most moment is the business of natural science, and that happiness depends on this (\videlicet{} on the knowledge of celestial and atmospheric phenomena), and upon knowing what the heavenly bodies really are, and any kindred facts contributing to exact knowledge in this respect.

Further, we must recognize on such points as this no plurality of causes or contingency, but must hold that nothing suggestive of conflict or disquiet is compatible with an immortal and blessed nature.
And the mind can grasp the absolute truth of this.

But when we come to subjects for special inquiry, there is nothing in the knowledge of risings and settings and solstices and eclipses and all kindred subjects that contributes to our happiness; but those who are well-informed about such matters and yet are ignorant what the heavenly bodies really are, and what are the most important causes of phenomena, feel quite as much fear as those who have no such special information – nay, perhaps even greater fear, when the curiosity excited by this additional knowledge cannot find a solution or understand the subordination of these phenomena to the highest causes.

Hence, if we discover more than one cause that may account for solstices, settings and risings, eclipses and the like, as we did also in particular matters of detail, we must not suppose that our treatment of these matters fails of accuracy, so far as it is needful to ensure our tranquillity and happiness.
When, therefore, we investigate the causes of celestial and atmospheric phenomena, as of all that is unknown, we must take into account the variety of ways in which analogous occurrences happen within our experience; while as for those who do not recognize the difference between what is or comes about from a single cause and that which may be the effect of any one of several causes, overlooking the fact that the objects are only seen at a distance, and are moreover ignorant of the conditions that render, or do not render, peace of mind impossible – all such persons we must treat with contempt.
If then we think that an event could happen in one or other particular way out of several, we shall be as tranquil when we recognize that it actually comes about in more ways than one as if we knew that it happens in this particular way.

There is yet one more point to seize, namely, that the greatest anxiety of the human mind arises through the belief that the heavenly bodies are blessed and indestructible, and that at the same time they have volitions and actions and causality inconsistent with this belief; and through expecting or apprehending some everlasting evil, either because of the myths, or because we are in dread of the mere insensibility of death, as if it had to do with us; and through being reduced to this state not by conviction but by a certain irrational perversity, so that, if men do not set bounds to their terror, they endure as much or even more intense anxiety than the man whose views on these matters are quite vague.
But mental tranquillity means being released from all these troubles and cherishing a continual remembrance of the highest and most important truths.

Hence we must attend to present feelings and sense perceptions, whether those of mankind in general or those peculiar to the individual, and also attend to all the clear evidence available, as given by each of the standards of truth.
For by studying them we shall rightly trace to its cause and banish the source of disturbance and dread, accounting for celestial phenomena and for all other things which from time to time befall us and cause the utmost alarm to the rest of mankind.

Here then, \sn{Herodotus}, you have the chief doctrines of Physics in the form of a summary.
So that, if this statement be accurately retained and take effect, a man will, I make no doubt, be incomparably better equipped than his fellows, even if he should never go into all the exact details.
For he will clear up for himself many of the points which I have worked out in detail in my complete exposition; and the summary itself, if borne in mind, will be of constant service to him.

It is of such a sort that those who are already tolerably, or even perfectly, well acquainted with the details can, by analysis of what they know into such elementary perceptions as these, best prosecute their researches in physical science as a whole; while those, on the other hand, who are not altogether entitled to rank as mature students can in silent fashion and as quick as thought run over the doctrines most important for their peace of mind.
\end{document}
