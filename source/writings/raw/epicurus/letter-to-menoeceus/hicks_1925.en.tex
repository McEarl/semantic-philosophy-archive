\documentclass{stex}
\libinput{preamble}
\begin{document}
\usemodule{glossary/persons?menoeceus}
\usemodule{glossary/persons?epicurus}
\usemodule{glossary/persons?theognis-of-megara}
\usemodule{glossary/places?hades}
\usemodule{glossary/writings/theognis-of-megara?elegies}

\noindent \sn{Epicurus} to \sn{Menoeceus}, greeting.

\vspace{1em}
Let no one be slow to seek wisdom when he is young nor weary in the search thereof when he is grown old.
For no age is too early or too late for the health of the soul.
And to say that the season for studying philosophy has not yet come, or that it is past and gone, is like saying that the season for happiness is not yet or that it is now no more.
Therefore, both old and young ought to seek wisdom, the former in order that, as age comes over him, he may be young in good things because of the grace of what has been, and the latter in order that, while he is young, he may at the same time be old, because he has no fear of the things which are to come.
So we must exercise ourselves in the things which bring happiness, since, if that be present, we have everything, and, if that be absent, all our actions are directed toward attaining it.

Those things which without ceasing I have declared unto thee, those do, and exercise thyself therein, holding them to be the elements of right life.
First believe that God is a living being immortal and blessed, according to the notion of a god indicated by the common sense of mankind; and so believing, thou shalt not affirm of him aught that is foreign to his immortality or that agrees not with blessedness, but shalt believe about him whatever may uphold both his blessedness and his immortality.
For verily there are gods, and the knowledge of them is manifest; but they are not such as the multitude believe, seeing that men do not steadfastly maintain the notions they form respecting them.
Not the man who denies the gods worshipped by the multitude, but he who affirms of the gods what the multitude believes about them is truly impious.
For the utterances of the multitude about the gods are not true preconceptions but false assumptions; hence it is that the greatest evils happen to the wicked and the greatest blessings happen to the good from the hand of the gods, seeing that they are always favourable to their own good qualities and take pleasure in men like unto themselves, but reject as alien whatever is not of their kind.

Accustom thyself to believe that death is nothing to us, for good and evil imply sentience, and death is the privation of all sentience; therefore a right understanding that death is nothing to us makes the mortality of life enjoyable, not by adding to life an illimitable time, but by taking away the yearning after immortality.
For life has no terrors for him who has thoroughly apprehended that there are no terrors for him in ceasing to live.
Foolish, therefore, is the man who says that he fears death, not because it will pain when it comes, but because it pains in the prospect.
Whatsoever causes no annoyance when it is present, causes only a groundless pain in the expectation.
Death, therefore, the most awful of evils, is nothing to us, seeing that, when we are, death is not come, and, when death is come, we are not.
It is nothing, then, either to the living or to the dead, for with the living it is not and the dead exist no longer. But in the world, at one time men shun death as the greatest of all evils, and at another time choose it as a respite from the evils in life.
The wise man does not deprecate life nor does he fear the cessation of life.
The thought of life is no offence to him, nor is the cessation of life regarded as an evil.
And even as men choose of food not merely and simply the larger portion, but the more pleasant, so the wise seek to enjoy the time which is most pleasant and not merely that which is longest.
And he who admonishes the young to live well and the old to make a good end speaks foolishly, not merely because of the desirableness of life, but because the same exercise at once teaches to live well and to die well.
Much worse is he who says that \textquote[\sn{Theognis of Megara}, \sn{Elegies}]{it were good not to be born, but when once one is born to pass with all speed through the gates of \sn{Hades}}.
For if he truly believes this, why does he not depart from life? It were easy for him to do so, if once he were firmly convinced.
If he speaks only in mockery, his words are foolishness, for those who hear believe him not.

We must remember that the future is neither wholly ours nor wholly not ours, so that neither must we count upon it as quite certain to come nor despair of it as quite certain not to come.

We must also reflect that of desires some are natural, others are groundless; and that of the natural some are necessary as well as natural, and some natural only.
And of the necessary desires some are necessary if we are to be happy, some if the body is to be rid of uneasiness, some if we are even to live.
He who has a clear and certain understanding of these things will direct every preference and aversion toward securing health of body and tranquillity of mind, seeing that this is the sum and end of a blessed life.
For the end of all our actions is to be free from pain and fear, and, when once we have attained all this, the tempest of the soul is laid; seeing that the living creature has no need to go in search of something that is lacking, nor to look for anything else by which the good of the soul and of the body will be fulfilled.
When we are pained because of the absence of pleasure, then, and then only, do we feel the need of pleasure.
Wherefore we call pleasure the alpha and omega of a blessed life.
Pleasure is our first and kindred good.
It is the starting-point of every choice and of every aversion, and to it we come back, inasmuch as we make feeling the rule by which to judge of every good thing.
And since pleasure is our first and native good, for that reason we do not choose every pleasure whatsoever, but ofttimes pass over many pleasures when a greater annoyance ensues from them.
And ofttimes we consider pains superior to pleasures when submission to the pains for a long time brings us as a consequence a greater pleasure.
While therefore all pleasure because it is naturally akin to us is good, not all pleasure is choiceworthy, just as all pain is an evil and yet not all pain is to be shunned.
It is, however, by measuring one against another, and by looking at the conveniences and inconveniences, that all these matters must be judged.
Sometimes we treat the good as an evil, and the evil, on the contrary, as a good.
Again, we regard independence of outward things as a great good, not so as in all cases to use little, but so as to be contented with little if we have not much, being honestly persuaded that they have the sweetest enjoyment of luxury who stand least in need of it, and that whatever is natural is easily procured and only the vain and worthless hard to win.
Plain fare gives as much pleasure as a costly diet, when once the pain of want has been removed,
while bread and water confer the highest possible pleasure when they are brought to hungry lips.
To habituate one's self, therefore, to simple and inexpensive diet supplies all that is needful for health, and enables a man to meet the necessary requirements of life without shrinking, and it places us in a better condition when we approach at intervals a costly fare and renders us fearless of fortune.

When we say, then, that pleasure is the end and aim, we do not mean the pleasures of the prodigal or the pleasures of sensuality, as we are understood to do by some through ignorance, prejudice, or wilful misrepresentation.
By pleasure we mean the absence of pain in the body and of trouble in the soul.
It is not an unbroken succession of drinking-bouts and of revelry, not sexual love, not the enjoyment of the fish and other delicacies of a luxurious table, which produce a pleasant life; it is sober reasoning, searching out the grounds of every choice and avoidance, and banishing those beliefs through which the greatest tumults take possession of the soul.
Of all this the beginning and the greatest good is prudence.
Wherefore prudence is a more precious thing even than philosophy; from it spring all the other virtues, for it teaches that we cannot lead a life of pleasure which is not also a life of prudence, honour, and justice; nor lead a life of prudence, honour, and justice, which is not also a life of pleasure.
For the virtues have grown into one with a pleasant life, and a pleasant life is inseparable from them.

Who, then, is superior in thy judgement to such a man? He holds a holy belief concerning the gods, and is altogether free from the fear of death.
He has diligently considered the end fixed by nature, and understands how easily the limit of good things can be reached and attained, and how either the duration or the intensity of evils is but slight.
Destiny, which some introduce as sovereign over all things, he laughs to scorn, affirming rather that some things happen of necessity, others by chance, others through our own agency.
For he sees that necessity destroys responsibility and that chance or fortune is inconstant; whereas our own actions are free, and it is to them that praise and blame naturally attach.
It were better, indeed, to accept the legends of the gods than to bow beneath that yoke of destiny which the natural philosophers have imposed.
The one holds out some faint hope that we may escape if we honour the gods, while the necessity of the naturalists is deaf to all entreaties.
Nor does he hold chance to be a god, as the world in general does, for in the acts of a god there is no disorder; nor to be a cause, though an uncertain one, for he believes that no good or evil is dispensed by chance to men so as to make life blessed, though it supplies the starting-point of great good and great evil.
He believes that the misfortune of the wise is better than the prosperity of the fool.
It is better, in short, that what is well judged in action should not owe its successful issue to the aid of chance.

Exercise thyself in these and kindred precepts day and night, both by thyself and with him who is like unto thee; then never, either in waking or in dream, wilt thou be disturbed, but wilt live as a god among men.
For man loses all semblance of mortality by living in the midst of immortal blessings.
\end{document}
