\documentclass{stex}
\usepackage[german]{babel}
\libinput{preamble}
\begin{document}
\begin{smodule}{letter-to-menoeceus_deu_1921_apelt}
\usemodule{glossary/abbreviations/de?vers}
\usemodule{glossary/abbreviations/de?das-ist}
\usemodule{glossary/persons?menoeceus}
\usemodule{glossary/places?hades}
\usemodule{writings/theognis-of-megara?elegies}

\metadata{
  source={1921_Apelt_LebenUndMeinungenBerühmterPhilosophen},
  license={CC0}
}

\begin{sparagraph}[id=letter-to-menoeceus,name=Letter to Menoeceus]
  \noindent \sr{Epicurus}{Epikur} entbietet dem \sr{Menoeceus}{Menoikeus} seine Gruß.

  \vspace{1em}
  Wer noch jung ist, der soll sich der Philosophie befleißigen, und wer alt ist, soll nicht müde werden zu philosophieren.
  Denn niemand kann früh genug anfangen für seine Seelengesundheit zu sorgen und für niemanden ist die Zeit dazu zu spät.
  Wer da sagt, die Stunde zum Philosophieren sei für ihn noch nicht erschienen oder bereits entschwunden, der gleicht dem, der behauptet, die Zeit für die Glückseligkeit sei noch nicht da oder nicht mehr da.
  Es gilt also zu philosophieren für jung und für alt, auf daß der eine auch im Alter noch jung bleibe auf Grund des Guten, das ihm durch des Schicksals Gunst zuteil geworden, der andere aber Jugend und Alter in sich vereinige dank der Furchtlosigkeit vor der Zukunft.
  Also gilt es unsern vollen Eifer dem zuzuwenden, was uns zur Glückseligkeit verhilft; denn haben wir sie, so haben wir alles, fehlt sie uns aber, so setzen wir alles daran, sie uns zu eigen zu machen.

  Wozu ich dich ohn’ Unterlaß mahnte, das mußt du auch tun und dir angelegen sein lassen, indem du dir klar machst, daß dies die Grundlehren sind für ein lobwürdiges Leben.
  Erstens halte Gott für ein unvergängliches und glückseliges Wesen, entsprechend der gemeinhin gültigen Gottesvorstellung, und dichte ihm nichts an, was entweder mit seiner Unvergänglichkeit unverträglich ist oder mit seiner Glückseligkeit nicht in Einklang steht; dagegen halte in deiner Vorstellung von ihm an allem fest, was danach angetan ist seine Glückseligkeit im Bunde mit seiner Unvergänglichkeit zu bekräftigen.
  Denn es gibt Götter, eine Tatsache, deren Erkenntnis einleuchtend ist; doch sind sie nicht von der Art, wie die große Menge sie sich vorstellt; denn diese bleibt sich nicht konsequent in ihrer Vorstellungsweise von ihnen.
  Gottlos aber ist nicht der, welcher mit den Göttern des gemeinen Volkes aufräumt, sondern der, welcher den Göttern die Vorstellungen des gemeinen Volkes andichtet.
  Denn was die gemeine Menge von den Göttern sagt, beruht nicht auf echten Begriffen, sondern auf wahrheitswidrigen Mutmaßungen.
  Daher läßt man den Bösen die größten Schädigungen von seiten der Götter widerfahren und den Guten die größten Wohltaten; denn ganz und gar für ihre eigenen Tugenden eingenommen, gönnen sie den Gleichgearteten alles Gute, während ihnen alles anders Geartete als fremdartig erscheint.

  Gewöhne dich auch an den Gedanken, daß es mit dem Tode für uns nichts auf sich hat.
  Denn alles Gute und Schlimme beruht auf Empfindung; der Tod aber ist die Aufhebung der Empfindung.
  Daher macht die rechte Erkenntnis von der Bedeutungslosigkeit des Todes für uns die Sterblichkeit des Lebens erst zu einer Quelle der Lust, indem sie uns nicht eine endlose Zeit als künftige Fortsetzung in Aussicht stellt, sondern dem Verlangen nach Unsterblichkeit ein Ende macht.
  Denn das Leben hat für den nichts Schreckliches, der sich wirklich klar gemacht hat, daß in dem Nichtleben nichts Schreckliches liegt.
  Wer also sagt, er fürchte den Tod, nicht etwa weil er uns Schmerz bereiten wird, wenn er sich einstellt, sondern weil er uns jetzt schon Schmerz bereitet durch sein dereinstiges Kommen, der redet ins Blaue hinein.
  Denn was uns, wenn es sich wirklich einstellt, nicht stört, das kann uns, wenn man es erst erwartet, keinen anderen als nur einen eingebildeten Schmerz bereiten.
  Das angeblich schaurigste aller Übel also, der Tod, hat für uns keine Bedeutung; denn so lange wir noch da sind, ist der Tod nicht da; stellt sich aber der Tod ein, so sind wir nicht mehr da.
  Er hat also weder für die Lebenden Bedeutung noch für die Abgeschiedenen, denn auf jene bezieht er sich nicht, diese aber sind nicht mehr da.
  Die große Menge indes scheut bald den Tod als das größte Aller Übel, bald sieht sie in ihm eine Erholung \textins{von den Mühseligkeiten des Lebens.
  Der Weise dagegen weist weder das Leben von sich} noch hat er Angst davor, nicht zu leben.
  Denn weder ist ihm das Leben zuwider noch hält er es für ein Übel, nicht zu leben.
  Wie er sich aber bei der Wahl der Speise nicht für die größere Masse, sondern für den Wohlgeschmack entscheidet, so kommt es ihm auch nicht daraf an, die Zeit in möglichster Länge, sondern in möglichst erfreulicher Fruchtbarkeit zu genießen.
  Wer aber den Jüngling auffordert zu einem lobwürdigen Leben, den Greis dagegen zu einem lobwürdigen Ende, der ist ein Tor nicht nur weil das Leben seine Annehmlichkeit hat, sondern auch, weil die Sorge für ein lobwürdiges Leben mit der für ein lobwürdiges Ende zusammenfällt.
  Noch weit schlimmer aber steht es mit dem, der da sagt, das Beste sei es, gar nicht geboren zu sein,
  \begin{displayquote}[\sr{verses 425--428}{Elegien, \vers 425--428}]
    aber, geboren einmal, sich schleunigst von dannen zu machen.
  \end{displayquote}
  Denn wenn er es mit dieser Äußerung wirklich ernst meint, warum scheidet er nicht aus dem Leben?
  Denn das stand ihm ja frei, wenn anders er zu einem festen Entschlusse gekommen wäre.
  Ist es aber bloßer Spott, so ist es übel angebrachter Unfug.
  Die Zukunft liegt weder ganz in unserer Hand noch ist sie völlig unserem Willen entzogen.
  Das ist wohl zu beachten, wenn wir nicht in den Fehler verfallen wollen, das Zukünftige entweder als ganz sicher anzusehen oder von vornherein an seinem Eintreten völlig zu verzweifeln.

  Zudem muß man bedenken, daß die Begierden teils natürlich teils nichtig sind und daß die natürlichen teils notwendig teils nur natürlich sind; die notwendigen hinwiederum sind notwendig teils zur Glückseligkeit teils zur Vermeidung körperlicher Störungen teils für das Leben selbst.
  Denn eine von Irrtum sich frei haltende Betrachtung dieser Dinge weiß jedes Wählen und jedes Meiden in die richtige Beziehung zu setzen zu unserer körperlichen Gesundheit und zur ungestörten Seelenruhe; denn das ist das Ziel des glückseligen Lebens.
  Liegt doch allen unseren Handlungen die Absicht zugrunde weder Schmerz zu empfinden noch außer Fassung zu geraten.
  Haben wir es aber einmal dahin gebracht, dann glätten sich die Wogen; es legt sich jeder Seelensturm, denn der Mensch braucht sich dann nicht mehr umzusehen nach etwas was ihm noch mangelt, braucht nicht mehr zu suchen nach etwas anderem, das dem Wohlbefinden seiner Seele und seines Körpers zur Vollendung verhilft.
  Denn der Lust sind wir dann benötigt, wenn wir das Fehlen der Lust schmerzlich empfinden; fühlen wir uns aber frei von Schmerz, so bedürfen wir der Lust nicht mehr.
  Eben darum ist die Lust, wie wir behaupten, Anfang und Ende des glückseligen Lebens.
  Denn sie ist, wie wir erkannten, unser erstes, angeborenes Gut, sie ist der Ausgangspunkt für alles Wählen und Meiden und auf sie gehen wir zurück, indem diese Seelenregung uns zur Richtschnur dient für Beurteulung jeglichen Gutes.
  Und eben weil sie das erste und angeborene Gut ist, entscheiden wir uns nicht schlechtweg für jede Lust, sondern es gibt Fälle, wor wir auf viele Annehmlichkeiten verzichten, sofern sich weiterhin aus ihnen ein Übermaß von Unannehmlichkeiten ergibt, und andererseits geben wie vielen Schmerzen vor Annehmlichkeiten den Vorzug, wenn uns aus dem längeren Ertragen von Schmerzen um so größere Lust erwächst.
  Jede Lust nun ist, weil sie etwas von Natur uns Angemessenes ist, ein Gut, doch nicht jede auch ein Gegenstand unserer Wahl, wie auch jeder Schmerz ein Übel ist, ohne daß jeder unter allen Umständen zu meiden wäre.
  Nur durch genaue Vergleichung und durch Beachtung des Zuträglichen und Unzuträglichen kann alles dies beurteilt werden.
  Denn zu gewissen Zeiten erweist sich das Gute für uns als Übel und umgekehrt das Übel als ein Gut.

  Auch die Genügsamkeit halten wir für ein großes Gut, nicht, um uns in jedem Falle mit Wenigem zu begnügen, sondern um, wenn wir nicht die Hülle und Fülle haben, uns mit dem Wenigen zufrieden zu geben in der richtigen Überzeugung, daß diejenigen den Überfluß mit der stärksten Lustwirkung genießen, die desselben am wenigsten bedürfen, und daß alles Naturgemäße leicht zu beschaffen, das Eitele aber schwer zu beschaffen ist.
  Denn eine bescheidene Mahlzeit bietet den gleichen Genuß wie eine prunkvolle Tafel, wenn nur erst das schmerzhafte Hungergefühl beseitigt ist.
  Und Brot und Wasser gewähren den größten Genuß, wenn wirkliches Bedürfnis der Grund ist sie zu sich zu nehmen.
  Die Gewöhnung also an eine einfache und nicht kostspielige Lebensweise ist uns nicht nur die Bürgschaft für volle Gesundheit, sondern sie macht den Menschen auch unverdrossen zur Erfüllung der notwenigen Anforderungen des Lebens, erhöht seine frohe Laune, wenn er ab und zu einmal auch einer Einladung zu kostbarerer Bewirtung folgt, und macht uns furchtlos gegen die Launen des Schicksals.
  Wenn wir also die Lust als das Endziel hinstellen, so meinen wir damit nicht die Lüste der Schlemmer und solche, die in nichts als dem Genusse selbst bestehen, wie manche Unkundige und manche Gegner oder auch absichtlich Mißverstehende meinen, sondern das Freisein von körperlichem Schmerz und von Störungen der Seelenruhe.
  Denn nicht Trinkgelage mit daran sich anschließenden tollen Umzügen machen das lustvolle Leben aus, auch nicht der Umgang mit schönen Knaben und Weibern, auch nicht der Genuß von Fischen und sonstigen Herrlichkeiten, die eine prunkvolle Tafel bietet, sondern eine nüchterne Verständigkeit, die sorgfältig den Gründen für Wählen und Meinden in jedem Falle nachgeht und mit allen Wahnvorstellungen bricht, die den Hauptgrund zur Störung der Seelenruhe abgeben.

  Für alles dies ist Anfang und wichtigstes Gut die vernünftige Einsicht, daher steht die Einsicht an Wert auch noch über der Philosophie.
  Aus ihr entspringen alle Tugenden.
  Sie lehrt, daß ein lustvolles Leben nicht mögliche ist ohne ein einsichtsvolles und sittliches und gerechtes Leben, und ein einsichtsvolles, sittliches und gerechtes Leben nicht ohne ein lustvolles.
  Denn die Tugenden sind mit dem lustvollen Leben auf das engste verwachsen, und das lustvolle Leben ist von ihnen untrennbar.
  Denn wer wäre deiner Meinung nach höher zu achten als der, der einem frommen Götterglauben huldigt und dem Tode jederzeit furchtlos ins Auge schaut?
  Der dem Endziel der Natur nachgedacht hat und sich klar darüber ist, daß im Reiche des Guten das Ziel sehr wohl zu erreichen und in unsere Gewalt zu bringen ist, und daß die schlimmsten Übel nur kurzdauernden Schmerz mit sich führen?
  Der über das von gewissen Philosophen als Herrin über alles eingeführte allmächtige Verhängnis lacht und vielmehr behauptet, daß einiges zwar infolge der Notwendigkeit entstehe, anderes dagegen infolge des Zufalls und noch anderes durch uns selbst; denn die Notwendigkeit herrscht unumschränkt, während der Zufall unstet und unser Wille frei (herrenlos, \dasIst nicht vom Schicksal abhängig) ist, da ihm sowohl Tadel wie Lob folgen kann.
  (Denn es wäre besser, sich dem Mythos von den Göttern anzuschließen als sich zum Sklaven der unbedingten Notwendigkeit der Physiker zu machen; denn jener Mythos läßt doch der Hoffnung Raum auf Erhörung durch die Götter als Belohnung für die ihnen erwiesene Ehre, diese Notwendigkeit dagegen ist unerbittlich.)
  Den Zufall aber hält der Weise weder für eine Gottheit, wie es der großen Menge gefällt (denn Ordnungslosigkeit verträgt sich nicht mit der Handlungsweise der Gottheit) noch auch für eine unstete Ursache (denn wer glaubt zwar, daß aus seiner Hand Gutes oder Schlimmes zu dem glücklichen Leben der Menschen beigetragen werde, daß aber von ihm nicht der Grund gelegt werde zu einer erheblichen Fülle des Guten oder des Schlimmen), denn er hält es für besser, bei hellem Verstande von Unglück verfolgt als bei Unverstand vom Glücke begünstigt zu sein.
  Das beste freilich ist es, wenn bei den Handlungen richtiges Urteil und glückliche Umstände sich zu gutem Erfolge vereinigen.

  Dies und dem Verwandtes laß dir Tag und Nacht durch den Kopf gehen und ziehe auch deinesgleichen zu diesen Überlegungen hinzu, dann wirst du weder wachend noch schlafend dich beunruhigt fühlen, wirst vielmehr wie ein Gott unter Menschen leben.
  Denn keinem sterblichen Wesen gleicht \emph{der} Mensch, der inmitten unsterblicher Güter lebt.
\end{sparagraph}
\end{smodule}
\end{document}
